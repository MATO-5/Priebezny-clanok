% Metódy inžinierskej práce

\documentclass[10pt,twocolumn,twoside,slovak,a4paper]{article}

\usepackage[slovak]{babel}
%\usepackage[T1]{fontenc}
\usepackage[IL2]{fontenc} % lepšia sadzba písmena Ľ než v T1
\usepackage[utf8]{inputenc}
\usepackage{graphicx}
\usepackage{url} % príkaz \url na formátovanie URL
\usepackage{hyperref} % odkazy v texte budú aktívne (pri niektorých triedach dokumentov spôsobuje posun textu)

\usepackage{cite}
%\usepackage{times}
\usepackage{subcaption}
\usepackage{graphicx} 
\usepackage{amsmath}
\usepackage{wrapfig}
\pagestyle{headings}

\title{Názov\thanks{Semestrálny projekt v predmete Metódy inžinierskej práce, ak. rok 2015/16, vedenie: Meno Priezvisko}} % meno a priezvisko vyučujúceho na cvičeniach

\author{Meno Priezvisko\\[2pt]
	{\small Slovenská technická univerzita v Bratislave}\\
	{\small Fakulta informatiky a informačných technológií}\\
	{\small \texttt{...@stuba.sk}}
	}

\date{\small 30. september 2015} % upravte



\begin{document}

\maketitle

\begin{abstract}
\ldots
\end{abstract}



\section{Úvod}

Motivujte čitateľa a vysvetlite, o čom píšete. Úvod sa väčšinou nedelí na časti.

Uveďte explicitne štruktúru článku. Tu je nejaký príklad.
Základný problém, ktorý bol naznačený v úvode, je podrobnejšie vysvetlený v časti~\ref{nejaka}.
Dôležité súvislosti sú uvedené v častiach~\ref{dolezita} a~\ref{dolezitejsia}.
Záverečné poznámky prináša časť~\ref{zaver}.

\section{prva cast}
toto je prva cast tohto clanku
\section{druha cast}
druha cast clanku

\section{Nejaká časť} \label{nejaka}
\section{Dôležitá časť} \label{dolezita}

V tejto časti uvádzame vývojový diagram, ktorý znázorňuje postup pri návrhu projektu.

\begin{figure}[h!]
    \centering
    \includegraphics[scale=0.5]{umlflowchart.pdf}
    \caption{Vývojový diagram projektu (UML štýl).}
    \label{fig:uml}
\end{figure}

Nasledujúci obrázok zobrazuje podobný postup, ale vytvorený v inom grafickom nástroji.

\begin{figure}[h!]
    \centering
    \includegraphics[scale=0.3]{flowchart.pdf}
    \caption{Vývojový diagram vytvorený v inom nástroji.}
    \label{fig:graphic}
\end{figure}
Z obr.~\ref{f:obrazok.png} je všetko jasné. 

\begin{figure*}[t!]
\centering
\includegraphics[width=\textwidth]{obrazok.png}
\caption{Rozhodujúci argument (cez oba stĺpce).}
\label{f:obrazok}
\end{figure*}

\begin{figure}[h!]
    \centering
    \begin{subfigure}[t]{0.48\columnwidth}
        \centering
        \includegraphics[width=\linewidth]{diagram1.pdf}
        \caption{Prvý UML diagram}
        \label{fig:diagram1}
    \end{subfigure}
    \hfill
    \begin{subfigure}[t]{0.48\columnwidth}
        \centering
        \includegraphics[width=\linewidth]{diagram3.pdf}
        \caption{Druhý UML diagram}
        \label{fig:diagram2}
    \end{subfigure}
    \caption{Dva UML diagramy vedľa seba}
    \label{fig:uml-diagrams}
\end{figure}
\begin{figure*}[t!]
    \centering
    \includegraphics[width=\textwidth]{obrazokcez2.jpg}
    \caption{Prehľadný UML diagram pre celý systém – cez oba stĺpce.}
    \label{fig:uml-wide}
\end{figure*}

\section{Iná časť} \label{ina}

Základným problémom je teda\ldots{} Najprv sa pozrieme na nejaké vysvetlenie (časť~\ref{ina:nejake}), a potom na ešte nejaké (časť~\ref{ina:nejake}).\footnote{Niekedy môžete potrebovať aj poznámku pod čiarou.}

Môže sa zdať, že problém vlastne nejestvuje\cite{Coplien:MPD}, ale bolo dokázané, že to tak nie je~\cite{Czarnecki:Staged, Czarnecki:Progress}. Napriek tomu, aj dnes na webe narazíme na všelijaké pochybné názory\cite{PLP-Framework}. Dôležité veci možno \emph{zdôrazniť kurzívou}.


\subsection{Nejaké vysvetlenie} \label{ina:nejake}

Niekedy treba uviesť zoznam:


\begin{wrapfigure}{l}{0.25\textwidth}  % l = left, šírka 25% stĺpca
    \vspace{-10pt}  
    \includegraphics[width=0.25\textwidth]{logo_fiit.png}
    \vspace{-10pt}  
\end{wrapfigure}

Toto je text

\begin{itemize}
\item jedna vec
\item druhá vec
	\begin{itemize}
	\item x
	\item y
	\end{itemize}
\end{itemize}


\begin{enumerate}
\item jedna vec
\item druhá vec
	\begin{enumerate}
	\item x
	\item y
	\end{enumerate}
\end{enumerate}


\subsection{Ešte nejaké vysvetlenie} \label{ina:este}

\paragraph{Veľmi dôležitá poznámka.}
Niekedy je potrebné nadpisom označiť odsek. Text pokračuje hneď za nadpisom.

\section{Príklad matice a dlhého vzorca}

Nasleduje príklad matice rozmeru 5x4:

\section{Príklad matice a dlhého vzorca}

% Matica cez oba stĺpce
\begin{figure*}[t!]
\[
A = 
\begin{bmatrix}
a_{11} & a_{12} & a_{13} & a_{14} \\
a_{21} & a_{22} & a_{23} & a_{24} \\
a_{31} & a_{32} & a_{33} & a_{34} \\
a_{41} & a_{42} & a_{43} & a_{44} \\
a_{51} & a_{52} & a_{53} & a_{54}
\end{bmatrix}
\]
\caption{Príklad matice 5x4 – zaberá oba stĺpce.}
\label{fig:matrix}
\end{figure*}

% Dlhý vzorec cez oba stĺpce
\begin{figure*}[t!]
\[
x_n = \sum_{k=0}^{n} \frac{(-1)^k}{k!} \Bigg( \sum_{i=0}^{k} \binom{k}{i} a_i b_{k-i} c_{i+k} d_{k-i+1} e_{i+k-1} f_{k} g_{i} h_{k-i} \Bigg)
\]
\caption{Príklad dlhého vzorca presahujúceho šírku stĺpca.}
\label{fig:long-formula}
\end{figure*}


\section{Dôležitá časť} \label{dolezita}




\section{Ešte dôležitejšia časť} \label{dolezitejsia}




\section{Záver} \label{zaver} % prípadne iný variant názvu



%\acknowledgement{Ak niekomu chcete poďakovať\ldots}


% týmto sa generuje zoznam literatúry z obsahu súboru literatura.bib podľa toho, na čo sa v článku odkazujete
\bibliography{literatura}
\bibliographystyle{plain} % prípadne alpha, abbrv alebo hociktorý iný
\end{document}
